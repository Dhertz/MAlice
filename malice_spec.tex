\documentclass[a4, 11pt]{article}

\usepackage{a4, fullpage}
\usepackage[tmargin=2.5cm, lmargin=2.5cm, rmargin=2.5cm, bmargin=2.5cm]{geometry}
\usepackage{titlesec}
\usepackage{enumitem}

\titlespacing\section{0pt}{11pt}{-2pt}
\titlespacing\subsection{0pt}{2pt}{-2pt}

\setlength{\parskip}{0.3cm}
\setlength{\parindent}{0cm}
\setlist{nolistsep}

\begin{document}

\title{\huge{MAlice Language Specification}}
\author{Owen Davies\\Daniel Hertz\\Charlie Hothersall-Thomas}
\date{\today}

\maketitle

\section*{Introduction}
This document outlines the specification of the MAlice language, which we have 
written based on the sample programs and their corresponding output which we 
were given. We will look at the lexical structure of the language, give a 
somewhat formal defenition of the lanugage in Backaus-Naur Form, and discuss 
any semantic decisions we have made about the language.

\section*{Lexical Structure}
\subsection*{Types}
There are two primative types in the MAlice language:

\begin{itemize}
  \item \textbf{number} - a number is a two's complement 32 bit integer 
  primative, with a range of \( -(2^{31}) \) to \( +(2^{31} - 1) \). 
  \item \textbf{letter} - a letter is an 8 bit character primative, which 
  allows scope for any character from the ASCII set.
\end{itemize}

\subsection*{Reserved Keywords}
The following table shows the list of keywords that cannot be used within a 
program, as they are keywords in the MAlice language.

\begin{center}
  \begin{tabular}{l l l l}
  was    & a   & number                                   & and \\
  became & but & \textquoteleft said Alice\textquoteright & too \\
  drank  & ate & letter                                   & then \\
  \end{tabular}
\end{center}

\subsection*{White Space}
White space is completely ignored and spacing is not important. Lines of MAlice 
code are seperated by delimitors - see the \textbf{Delimiters} section for more 
information on this.

\subsection*{Operators}
This table shows the operators that can be used in the language. It also shows 
the level of precedence of the operator.

\begin{center}
  \begin{tabular}{| c | c | c |}
  \hline
  \textbf{Operation}&\textbf{Symbol}&\textbf{Precedence Level}\\
  \hline
  Addition         & +                     & 2 \\
  Bitwise AND      & \&                    & 3 \\
  Bitwise NOT      & $\sim$                & 0 \\
  Bitwise OR       & \textbar              & 4 \\
  Bitwise XOR      & \textasciicircum      & 5 \\
  Integer Division & /                     & 1 \\
  Modulus          & \%                    & 1 \\
  Multiplication   & \textasteriskcentered & 1 \\
  Subtraction      & \textendash           & 2 \\
  \hline
  \end{tabular}
\end{center}

\section*{Semantic Decisions}
\subsection*{Delimiters}
\subsection*{Variable Name Constraints}

\section*{Backaus-Naur Form Grammar}

\end{document}
