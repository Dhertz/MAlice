\documentclass[a4, 11pt]{article}

\usepackage{a4, fullpage}
\usepackage[tmargin=2.5cm, lmargin=2.5cm, rmargin=2.5cm, bmargin=2.5cm]{geometry}
\usepackage{titlesec}
\usepackage{enumitem}
\usepackage{textcomp}

\titlespacing\section{0pt}{11pt}{-2pt}
\titlespacing\subsection{0pt}{2pt}{-2pt}

\setlength{\parskip}{8pt}
\setlength{\parindent}{0pt}
\setlist{nolistsep}

\begin{document}

\title{\huge{MAlice Language Specification}}
\author{Owen Davies\\Daniel Hertz\\Charlie Hothersall-Thomas}
\date{\today}

\maketitle

\section*{Introduction}
This document outlines the specification of the MAlice language, which we have written based on the provided sample programs and their corresponding outputs. We will look at the lexical structure of the language, give a somewhat formal definition of the language in Backaus-Naur Form, and discuss any semantic decisions we have made about the language.

\section*{Lexical Structure}
\subsection*{Types}
There are two primitive types in the MAlice language:

\begin{itemize}
  \item \textbf{number} - a number is a two's complement 32-bit integer 
  primitive, with a range of \( -(2^{31}) \) to \( +(2^{31} - 1) \). 
  \item \textbf{letter} - a letter is an 8 bit character primitive, which 
  allows scope for any character from the ASCII set.
\end{itemize}

\subsection*{Reserved Keywords}
The following table shows the list of keywords that cannot be used within a program, as they are keywords in the MAlice language:

\begin{itemize}
  \item \textbf{and} - this keyword is a statement separator. See the \textbf{Delimiters} section.
  \item \textbf{ate} - this keyword increments a previously declared variable.
  \item \textbf{became} - this keyword allows assignment to a declared variable.
  \item \textbf{but} - see \textbf{and}.
  \item \textbf{drank} - this keyword decrements a previously declared variable.
  \item \textbf{letter} - this is a primitive. See the \textbf{Types} section above.
  \item \textbf{number} - this is a primitive. See the \textbf{Types} section above.
  \item \textbf{said} - this is used along with \textquoteleft Alice\textquoteright{} to output values.
  \item \textbf{then} - see \textbf{and}.
  \item \textbf{too} - this keyword can follow a variable declaration as long as a variable has been declared directly before it and that variable was of the same type as the variable currently being declared.
  \item \textbf{was} - this keyword is used along with \textquoteleft \textbf{a}\textquoteright{} (and a type) to declare variables.
\end{itemize}

\subsection*{White Space}
White space is completely ignored and spacing is not important. Lines of MAlice code are separated by delimiters - see the \textbf{Delimiters} section for more information on this.

\subsection*{Operators}
This table shows the operators that can be used in the language. It also shows the level of precedence of the operator. All operators are binary, apart from \textbf{Bitwise NOT} which is unary (and written before the operand).

\begin{center}
  \begin{tabular}{| c | c | c |}
  \hline
  \textbf{Operation}&\textbf{Symbol}&\textbf{Precedence Level}\\
  \hline
  Addition         & +                     & 2 \\
  Bitwise AND      & \&                    & 3 \\
  Bitwise NOT      & \( \sim \)            & 0 \\
  Bitwise OR       & \textbar              & 4 \\
  Bitwise XOR      & \( \wedge \)          & 5 \\
  Integer Division & \textfractionsolidus  & 1 \\
  Modulus          & \%                    & 1 \\
  Multiplication   & \textasteriskcentered & 1 \\
  Subtraction      & \textminus            & 2 \\
  \hline
  \end{tabular}
\end{center}

\section*{Semantic Decisions}
\subsection*{Delimiters}
There are 2 main categories of delimiters in MAlice - ones that end lines, and ones that combine multiple lines into one. Full stops and commas separate lines, while the keywords \textquoteleft \textbf{and}\textquoteright , \textquoteleft \textbf{but}\textquoteright{} and \textquoteleft \textbf{then}\textquoteright{} connect two lines together. As whitespace is removed at compile time, there can be as many delimiters as necessary on a single line in the source file.

\subsection*{Variable Name Constraints}
Variable names must begin with either a lower- or upper-case letter. Underscores are allowed, provided that they are in the middle of the variable name (i.e. not the first or last character). Capital letters can follow an underscore character, but all other characters must be lower case. These constraints have been modelled on the English language, and can be represented with the below regular expression:

\begin{center}
  \texttt{[A-Za-z](\_[A-Za-z]\textbar [a-z])*}
\end{center}

\section*{Backaus-Naur Form Grammar}

\end{document}
